\documentclass[12pt, a4paper]{article}
  \usepackage[brazilian]{babel}
  \usepackage{graphicx,color}
  \graphicspath{{../figuras/}}
  \newtheorem{teo}{Teorema}[section]
  \newtheorem{lema}[teo]{Lema}
  \newtheorem{cor}[teo]{Corol\'ario}
  \newtheorem{prop}[teo]{Proposiç\~ao}
  \usepackage[utf8]{inputenc}
  \usepackage{csquotes}
  \usepackage[backend=biber, style=numeric]{biblatex}
  \addbibresource{kopka-daly.bib}
\begin{document}
  \tableofcontents
  \title{texto do t\'itulo}
  \author{autor1 \cite{TESTE} \\endereço1 \cite{kopka-daly} \and autor2\\endereço2}
  \printbibliography
  \maketitle
    \subsection{t\'itulo} \subsubsection{t\'itulo}
  \begin{center}
    linha 1 \\ linha 2 \\ linha n
  \end{center}
  \begin{quote}
    Texto a ser indentado.
  \end{quote}
  \begin{itemize}
    \item Os itens
    s\~ao precedidos por $\bullet$;
    \item Os itens
    s\~ao separados por um espaço adicional.
  \end{itemize}
  \begin{enumerate}
    \item Os itens
    s\~ao numerados com algarismos arábicos, no primeiro n\'ivel,
  \begin{enumerate}
    \item
    s\~ao numerados com letras no segundo n\'ivel e
  \begin{enumerate}
    \item
    s\~ao numerados com algarismos romanos no terceiro n\'ivel.
  \end{enumerate}
  \end{enumerate}
  \end{enumerate}
  \begin{teo}[Pitágoras] 
  Em todo tri\^angulo ret\^angulo o quadrado do comprimento da hipotenusa \'e igual a soma dos quadrados dos comprimentos dos catetos.
\end{teo}
\end{document}